\documentclass[11pt,twocolumn]{article}
\usepackage{graphicx}
\begin{document}

\title{LISA release mechanism}
\author{A. Collogrosso, D. Cuzzocrea, A. Mastrantuono}
\date{\today}
\maketitle

\section{INTRODUCTION}
LISA (Laser Interferometer Space Antenna) is a science misson aimed at revealing gravitational waves by means of the formation distortion of 3 \textit{test masses} (TMs), each one in a different spacecraft, put in free-falling (geodesic) trajectories. Due to weak gravitational waves interaction, any force different from gravitational ones acting on TMs must be negligible, and TM release must meet the requirements listed in tab.1:

\begin{table}[h]
\caption{requirements for TM release}
\begin{tabular}{l|l} \hline
parameter & tolerance \\ \hline
offset along x, y and z & $\pm 200 \mu m$\\
linear velocity along x, y and z & $\pm 5 \mu m/s$\\
angle around x, y and z & $\pm 2 mrad$\\
angular velocity around x, y, z& $\pm 100 \mu rad/s$\\ 
\end{tabular}
\end{table}

To meet these strict requirments the \textit{Caging Mechanism}(CM) have been designed as composed of two subsystems: \textit{Caging Mechanism Sub System} (CMSS), which locks the TM during launch and orbital transfer by appling forces up to 3000 N, and the \textit{Grabbing Positioning and Release System} (GPRS), object of this report, which sets the TM into high precision geodesic trajectory. Fig. 1 represents a scheme of GPRS. 

\begin{figure}[h]
\includegraphics[width=\linewidth]{GPRS-image.png}
\end{figure}

The release procedure is the following:

\begin{enumerate}
\item \textit{Grabbing finger} (GF) holds and positions the TM before release
\item \textit{Piezo-stack actuators} is activeted and pushes the \textit{Release tip} (RT) on the TM until force sensor measures a contact force of $0.2\div 3 N$. 
\item  GF is them moved back using the \textit{NEXLINE actuator}. This reduces RT force without losing contact with TM. Operations (2) and (3) are repested until GF has rached a distance of about 0.5 mm from TM. Force sensor indication$=0.2 \div 3 N$
\item The force is then reduced to the lowest acceptable value that still controls TM. This is done in order to minimize residual adhesion force between TM and the spherical contact surface of the RT.
\item TM release occurs by fast \textit{Piezo-stack actuators} retraction. The spring provides necessary force to break residual adhesion forces. A net momentum transfer takes place in this phase due to the low repeatibility of adhesion bonds and to possible delays in release.
\item Eventually GF is retracted to give clearance to TM for his electronic stabilization.
\end{enumerate}

\end{document}